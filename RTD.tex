\documentclass{article}

\usepackage{amsmath}
\usepackage{mathbbol}
\usepackage{hyperref}
\usepackage{verbatim}
\usepackage[margin=2cm]{geometry}

\newcommand{\oph}[2]{\ensuremath{|#1\rangle\langle#2|}}
\newcommand{\bra}[1]{\ensuremath{\langle #1 |}}
\newcommand{\ket}[1]{\ensuremath{|#1\rangle}}
\newcommand{\bral}[1]{\ensuremath{( #1 |}}
\newcommand{\ketl}[1]{\ensuremath{|#1)}}
\newcommand{\G}{\ensuremath{\mathcal{G}}}
\newcommand{\J}{\ensuremath{\mathcal{J}}}
\newcommand{\Tr}{\ensuremath{\text{Tr}}}
\newcommand{\Tru}[1]{\ensuremath{\underset{#1}{\text{Tr}}}}

\title{Real Time Diagrammatics in QmeQ}
\author{Martin Josefsson}
\begin{document}
\maketitle
\section{Introduction}
This document contains the theoretical background for the Real Time Diagrammatic (RTD) approach in QmeQ. The formal mathematics are presented along with detailed comments on how that is translated into the QmeQ code. The document does not contain any discussion about the physics that can be modeled using this approach, or any in-depth comparison between different master equation approaches. 
\subsection{Superoperator notation}
Central to the construction of the master equations using the RTD theory are Liouville superoperators, defined by their action on standard operators acting on some Hilbert space. The most common example is the commutator superoperator (referred to as Liouvillian)

\begin{equation}
    L\bullet = [H, \bullet]_- = H\bullet-\bullet H,
\end{equation}
where $H$ is a Hamiltonan and $\bullet$ is an arbitrary Hilbert space operator.  Since any superoperator $A$ is a map between pairs of operators ($O_1 = \sum_{a,b}o_{1,ab}|a\rangle\langle b|,  O_2 = \sum_{c,d}o_{2,cd}|c\rangle\langle d|)$, indexing with respect to Hilbert space states requires four indices for each element in $A$

\begin{equation}
	A^{ab}_{cd} = \bra{c} \Big ( A \oph{a}{b} \Big  )\ket{d}.
    \label{Eq:MatrixElement}
\end{equation}
\\
To simplify this notation we introduce the following Liouville states 

\begin{align}
	\ketl{ab} &= \oph{a}{b}, \\
	\bral{cd} &= \text{Tr}\oph{c}{d}^\dagger,
\end{align}
\\
allowing us to express the matrix elements of $A$ as 

\begin{equation}
	A^{ab}_{cd} = \bral{cd}A\ketl{ab}.
\end{equation}
\\
Furthermore, we can express the trace as $\bral{\mathbb{1}} = \text{Tr} \oph{i}{i}^\dagger$, the completeness relation as $1 = \sum_{ab}\ketl{ab}\bral{ab}$ and the density matrix $\rho$ as $\ketl{\rho}= \rho_{ij}\oph{i}{j}.$
\\
\\
In the RTD approach one defines superoperators analogous to the fermion field operator in Hilbert space (creation and annihilation) to be able to translate the Hamiltonian to a Liouvillian form that closely resembles the initial Hamiltonian. These fermion field operators have taken different forms in in different publications (see Refs.~\cite{leijnse2008kinetic, saptsov2012fermionic, gergs2015charge, gergs2017transport, gergs2018spin} and references therein), and the RTD approach in QmeQ uses results derived with different operator definitions for different kernels.
\\\\
The simplest version used is 
\begin{equation}
	G^{p_1}_1 \bullet = 
	\begin{cases}
		d_1\bullet,\ p_1=+ \\
		\bullet d_1,\ p_1=-
	\end{cases}
	\label{Eq:G1}
\end{equation}
\\
where $d$ is a field operator acting on the dot subspace. The number subscript (e.g. $1$) includes \textbf{all} relevant quantum numbers, which are an electron-hole index $\eta$:

\begin{equation}
	d_1 = 
	\begin{cases}
		d_{l_1}^\dagger, \ \eta=+\\
		d_{l_1}, \ \eta=-
	\end{cases},	
\end{equation}
\\
and an orbital index $l_1$ for a dot operator. For a reservoir operator, denoted by $c_1$, the subscript contains the wave vector/energy $k_1/\omega_1$, reservoir index $r_1$ and electron-hole index $\eta_1$.(Note that spin is not explicitly used in QmeQ since all reservoirs and orbitals are taken to be fully spin-polarized). Numbers on single electron tunnel matrix elements $t$ represent $r, \omega$, $\eta$ and $l$. A subscript with a bar on top means that either $\eta$ or $q$ (see blow) should be inverted based on the context.
\\
\\
The second kind of fermion superoperator we will use is defined as~\cite{saptsov2012fermionic}

\begin{equation}
	\mathcal{G}_1^{q_1}=\frac{1}{\sqrt{2}}\left(d_1\bullet  + q_1(-1)^{N_D}\bullet (-1)^{N_D}d_1\right),\ \ \ q_1\in \{+,-\},
	\label{Eq:G2}
\end{equation}
\\
where $N_D=\sum_l d^\dagger_l d_l$ is the number operator for the dot and $(-1)^{N_D}:=e^{i\pi N_D}=\prod_l (1-2N_l)$ the parity. The operators in Eq. \ref{Eq:G1} and Eq. \ref{Eq:G2} are related to one another as 

\begin{align}
	\mathcal{G}^q_1 = \frac{1}{\sqrt{2}}\sum_p p^{\frac{1-q}{2}}G_1^p p^{L_{N_D}} \\
	G^p_1 = \frac{1}{\sqrt{2}}\sum_q q^{\frac{1-p}{2}}\mathcal{G}_1^p p^{L_{N_D}}
\end{align}
\\
where the "superfermion" operator $L_{N_D}\bullet=[N_D, \bullet]_-$ counts the number of "superfermion" excitations to the right.  The corresponding reservoir superoperators are defined as 

\begin{equation}
	\mathcal{J}_1^{q_1}=\frac{1}{\sqrt{2}}\left(c_1\bullet  + q_1(-1)^{N_R}\bullet (-1)^{N_R}c_1\right),\ \ \ q_1\in \{+,-\},
	\label{Eq:J2}
\end{equation}
\\
with $N_R = \sum_{r, k}n^\dagger_{r,k}n_{r,k}$. This document will mostly focus on the $\mathcal{G}$ operators since the $G$ operators are only used in Ref.~\cite{leijnse2008kinetic} where they are well-documented.
\\
\\
For a diagonal density matrix we mostly need to evaluate matrix elements of $\mathcal{G}$ between a diagonal and a non-diagonal state

\begin{align}
	\bral{ab}\mathcal{G}^q_1\ketl{cc} &= \frac{1}{\sqrt{2}} \bra{a}d_1\ket{b}\big ( \delta_{cb} + q\delta_{ac}\big ) \\
	\bral{cc}\mathcal{G}^+_{\bar{1}}\ketl{ab} &= \bral{ab}\mathcal{G}^-_1\ketl{cc}^*.
    \label{Eq:MeG2}
\end{align}
\\
The $\mathcal{G}$ operators obey 

\begin{align}
	(\mathcal{G}_1^{q_1})^\dagger &= \mathcal{G}_{\bar{1}}^{\bar{q_1}} \\
	[\G_2^{q_2}, \G_1^{q_1}]_+ &= \delta_{2\bar{1}}\delta{q_2\bar{q_1}}\mathcal{I}
\end{align}
\\
with $\mathcal{I}$ denoting the identity matrix. Other important properties we we'll use later are $\bral{\mathbb{1}}\mathcal{G}^+=\bral{\mathbb{1}}\mathcal{J}^+=0$. 
%-----------------------------------------------------subsection System Liouvillian
\subsection{ System Liouvillian}
Before deriving the master equations one starts with translating the total system Hamiltonian to Liouville form. The system Hamiltonian is given by (assuming the QD Hamiltonian is diagonalized)

\begin{align}
	H &= H_R + H_T + H_D, \\
	H_D &= \sum_{a}\varepsilon_a \oph{a}{a} , \\
	H_R &= \sum_{k,r=L,R}\omega_{k,r}c^\dagger_{k,r}c_{k,r}, \\
	H_T &= \sum_{k,r,l} t_{k,r,l}d^\dagger_l c_{k,r} + H.c.
     \label{Eq:Hamiltonian}
\end{align}
\\
Here $\ket{a}$ denotes a many-body eigenstate of the QD Hamiltonian with energy $\varepsilon_a$. Using the fermion superoperators in Eqs. \ref{Eq:G2} and \ref{Eq:J2} we can rewrite the total Hamiltonian as

\begin{align}
	L &= L_R + L_T + L_D, \\
	L_D &= [H_D, \bullet]_-, \\
	L_R &= [H_R, \bullet]_-= \sum_1 \eta_1\omega_1\J^+_1J^-_{\bar{1}}, \\
	L_T &= [H_T, \bullet]_-=\sum_{1,r}t_{1}\eta_1\delta_{rr_1}\sum_{q_1}\G^{\bar{q_1}}_{1}\J^{q_1}_{\bar{1} }
\end{align}
\\
The tunnel matrix elements obey $t_{\bar{1}}=t_1^*$. Note  that $\bral{\mathbb{1}^R}L_R=0$ and $\bral{\mathbb{1}^D}L_D=0$.
%--------------------------------------------
\subsection{ Derivation of the master equations}

The time dependence of the density matrix is governed by the Liouville-von Neumann equation

\begin{equation}
    \dot{\rho} = -iL\rho,
\end{equation}
\\
which has the formal solution $\rho(t) = e^{-iL}\rho(0)$. We are interested in the stationary state solution for the reduced density matrix of the dot $\rho_D=\Tru{R}\rho$. In order for us to set up our master equations we assume that the reservoirs and dot are initially uncorrelated $\rho(0)=\rho_R\otimes\rho_D(0)$. We now start the derivation by Laplace transforming the time dependent solution for $\rho_D$

\begin{equation}
    \rho_D = \Tru{R}\int_0^\infty e^{izt}e^{-iLt}\rho(0)=\Tru{R}\frac{i}{z-L_D-L_R-L_T}\rho_D(t_0)\otimes\rho_R.
\end{equation}
\\
By expanding the quotient in $L_T$ one obtains the geometric series 

\begin{equation}
	\rho_D(z) = \left [ \frac{i}{z-L_D}\sum_{k=0}^{\infty}\left (W(z)\frac{i}{z-L_D}\right)^k\right]\rho_D(t_0),
	\label{Eq:Wseries}
\end{equation}
\\
where $\Tru{R}L_R=0$ was used. $W(z)$ contains a series of "irreducible diagrams" (see Refs.~\cite{leijnse2008kinetic, saptsov2012fermionic})

\begin{equation}
	W(z)=-i\sum_{n=1}^\infty\Tru{R}\left(L_T\frac{1}{z-L_D-L_R}\right )^nL_T\rho_R\Big|_{irred}.
	\label{Eq:irred}
\end{equation}
\\
Using the fact that Eq. \ref{Eq:Wseries} is a geometric series we can re-sum it and write 

\begin{align}
	&\rho_D(z) = \frac{i}{z-L_D-iW(z)}\rho_D(t_0), \\
	(z-&L_D - iW(z))(-iz)\rho_D(z)=z\rho_D(t_0).
\end{align}
\\
Now we use the fact that the stationary state density matrix is given by $\lim_{t\rightarrow \infty}\rho(t) = \lim_{z\rightarrow i0^+}(-iz)\rho(z)$ to take the zero frequency limit and obtain

\begin{equation}
	\left(-iL_D + W(i0^+)\right)\rho_D =0,
	\label{Eq:ME}
\end{equation}
\\
where $\rho_D$ without any argument denotes the stationary state density matrix. Equation \ref{Eq:ME} is exact as long as all terms in Eq. \ref{Eq:irred} are included. This is however not possible from a practical point of view. Therefore the series is usually truncated at order $L_T^2$ (sequential tunneling) or $L_T^4$ (cotunning order). Only terms of even $n$ in Eq. \ref{Eq:irred} give a non-vanishing contribution.
%---------------------------------------------------subsection ME in qmeq
\subsection{Master equations in QmeQ}

The addition of Real Time Diagrammatics in QmeQ provides the functionality to solve for the diagonal density matrix and currents at order $L_T^2 + L_T^4$. The kernel used for solving for the diagonal density matrix in that case is given by (see Ref.~\cite{leijnse2008kinetic})

\begin{equation}
	W = W_{dd}^{(1)} + W_{dd}^{(2)} + W_{dn}^{(1)}\left(L_{nn}\right)^{-1}W_{nd}^{(1)},
    \label{Eq:Fullkernel}
\end{equation}
\\
where we've dropped the argument $i0^+$. The first two terms are the kernels connecting diagonal ($\oph{a}{a}$) and diagonal ($\oph{b}{b}$) entries of the density matrix in first ($\propto L_T^2$) and second ($\propto L_T^4$) order. $W_{dd}^{(1)}$ is equivalent to the kernel in the Pauli approach. The last term adds off-diagonal corrections due to first order tunneling processes (the corrections are of order $\Gamma^2$). Without this term unphysical artifacts can arise in some situations. In this term $W_{dn/nd}^{(1)}$ are the parts of the full off-diagonal kernel in first order connecting diagonal ($\oph{a}{a}$) and non-diagonal ($\oph{b}{c}$) density matrix entries. $L_{nn}$ is the Liouvillian ($[H_D, \bullet]_-$) connecting non-diagonal and non-diagonal entries.
\\
\\
The correction term is included in all calculations by default, but can be turned off using 

\begin{verbatim}
    system = qmeq.Builder(..., kerntype='RTD') 
    system.off_diag_corrections = False 
    system.solve()
\end{verbatim}
The RTD approach supports solving the equation system $W\rho_D=\mathbf{0}$ using the normal QmeQ options \verb !solmethod='solve'! (LAPACK routine DGESV) and least squares \verb!solmethod='lsqr'! (LAPACK routine DGELSD). The RTD approach does not support matrix free solvers.
%-----------------------------------------------
\subsection{Observables}
\label{Sec:Observables}

The main output of the calculations in QmeQ are expectation values for different operators, namely the charge, energy and heat currents. The expectation value for a generic operator $A$ is given by 

\begin{equation}
    \langle A  \rangle(z) = \Tr \left [A \rho(z) \right ] = \frac{1}{2}\Tr \left [ L_A^+\rho(z) \right ], \ \ \ L_A^+=[A,\bullet]_+.
\end{equation}
\\
Using the laplace transform of the Liouville-von Neumann equation we rewrite this as 

\begin{equation}
    \langle A \rangle (z) =\Tr \left[ L_A^+\frac{i}{z-L_R-L_D-T} \rho(t_0)\right ],
    \label{Eq:FormalObs}
\end{equation}
\\
which we can evaluate by expanding in powers of $L_T$ and using the same diagrammatic rules as for the density matrix. However, this requires setting up new kernel(s) that can have very complicated integral expressions. In the RTD addition to QmeQ the philosophy is instead to use the already computed kernel (used to obtain $\rho_D$) to as large extent as possible. To avoid unnecessary kernel evaluations we use the approach described in section II.E of Ref.~\cite{saptsov2012fermionic}. The key to use that approach is that the observable we want to evaluate can be express as a commutator. This is always true in our case since we are evaluating time derivatives in the Heisenberg picture. This approach, however, only works for quantities that are conserved in a tunneling process, like the charge

\begin{equation}
    I_r = \langle -\frac{dN_r}{dt} \rangle = -i\langle [H, N_r] \rangle = -i\langle [H_{T,r}, N_r] \rangle = i\langle [H_{T,r}, N_D] \rangle, 
    \label{Eq:ChargeCons}
\end{equation}
\\
where the last step comes from $[N_r+N_D, H_{T,r}]=0$. Using Eq. \ref{Eq:ChargeCons} it is thus possible to reduce the number of reservoir operators in Eq. \ref{Eq:FormalObs} by one, simplifying the expressions resulting from taking the trace over the reservoirs. Furthermore, using the identity $[[A,B],\bullet]_+=[A,[B,\bullet]]_+$ it's possible to write

\begin{equation}
    L_{I_r}^+ = \frac{i}{2} \left (  L_{N_D}^+L_{T,r} - L_{T,r}L_{N_D}^+ \right ) 
\end{equation}
\\
where the last term can always be neglected since $\bral{\mathbb{1}}L_{T,r}\bullet=0$. Inserting into Eq. \ref{Eq:FormalObs} yields

\begin{equation}
    \langle I_r \rangle (z) = \frac{i}{2}\Tr \left[ L_{N_D}^+L_{T,r}\rho(z)\right ] = \frac{i}{2}\Tru{D} L_{N_D}^+ \left ( \Tru{R} \left[ L_{T,r}\frac{i}{z-L_R-L_D-T} \rho(t_0)\right ]\right ) = \frac{1}{2}\Tru{D}L_{N_D}^+W_r(z)\frac{i}{z-L_D-L_R-L_T}\rho(t_0)
\end{equation}
\\
in which $L_{N_D}^+$ did not contain any reservoir operators and could be pulled to the outside of the reservoir trace. In the stationary state limit $\lim_{t\rightarrow\infty} \langle I_r\rangle (t)= -iz\lim_{z\rightarrow i0^+}\langle I_r\rangle (z)$ this gives us

\begin{equation}
   \langle I_r\rangle= \frac{1}{2}\Tru{D}L_{N_D}^+ W_r(i0^+)\rho_D,
\end{equation}
\\
where we from hereon skip the argument $i0^+$. The kernel $W_r$ is similar to the full kernel $W$ with the exception that the left-most $L_T$ is replaced by a $L_{T,r}$. In QmeQ every $W_r$ is stored such that the charge current for each reservoir can be calculated as 

\begin{equation}
    \langle I_r\rangle = \frac{1}{2}\Tru{D}L_{N_D}^+ W_r\rho_D = \frac{1}{2}\bral{2N_D}W_r\ketl{\rho_D},
\end{equation}
\\
without having to calculate any new kernels or solve additional integrals. The full kernel is then obtained from  $W=\sum_r W_r$. 
\\
\\
A similar treatment is possible for the energy current $E_r$, but additional care is needed since $[H_{T,r}, H_r+H_D]\ne0$ and correction terms arise. These terms are described in detail later.
%---------------------------------------------------------------- section: Kernels
\section{Kernels}

As can be seen from Eq. \ref{Eq:Fullkernel}, solving the transport problem using the RTD approach requires evaluating quite a few (sub)kernels, in contrast to other techniques implemented in QmeQ. In addition to the kernels in Eq. \ref{Eq:Fullkernel} two more kernels are needed for the energy current. Most subkernels are lead-resolved, meaning that they store \verb!nlead*npauli*npauli! elements instead of just \verb!npauli*npauli! as in the Pauli approach. 
\\
\\
Below follows descriptions of the kernels used in the RTD approach along with comments about implementation details.
%----------- subsection: Wdd1
\subsection{$W^{(1)}_{dd}$}

Using the diagrammatic rules detailed in Refs.~\cite{leijnse2008kinetic, saptsov2012fermionic, gergs2017transport} we can write down all diagrams containing terms of order $L_T^2$, which is only a single one $\left( L_T\frac{1}{z-L_D-L_R}L_T\rho_R\right )$. Going from the diagrammatic expression to something useful requires commuting all reservoir operators $\mathcal{J}$ to the left using $L_R\J_1^q=\J_1^{q}(L_R+\eta_1\omega_1)$, contracting two $\mathcal{J}$s and tracing/integrating over the reservoir degrees of freedom: $\langle \mathcal{J}_1\mathcal{J}_2\rangle_R$. Using the operators defined in Eq. \ref{Eq:G2} the diagram gives (remember that the sum over 1 implicitly means summing over $\eta_1, q_1, l_1, r_1$) \cite{gergs2015charge, gergs2017transport, gergs2018spin}

\begin{equation}
    -i\int d\omega_1\sum_1\frac{\Gamma_{1\bar{2}}}{2\pi}\mathcal{G}_{\bar{2}}^+ \frac{q_1\gamma_1^{q_1}}{i0^+-L_D-\eta_1\omega_1}\mathcal{G}_{1}^{\bar{q_1}},
        \label{Eq:Wdd1}
\end{equation}
\\
where $\Gamma_{1\bar{2}} = 2\pi t_1t_{\bar{2}}$ (we assume that a factor $\sqrt{\nu_r}$ is included in $t$, $\nu_r$ being the density of states of reservoir $r$). The contraction function $\gamma_1^{q_1}$ is defined as 

\begin{equation}
    \gamma_1^{q_1} = \langle \J^-_{\bar{1}}\J^{q_1}_1\rangle_R = \delta_{q_1+} + \delta_{q_1-}\tanh\left ( \frac{\eta_1(\omega_1-\mu_1)}{2T_1} \right ).
\end{equation}
\\
The matrix $W^{(1)}_{dd}$ is built element by element (see Eq. \ref{Eq:MatrixElement}) using Eq. \ref{Eq:MeG2} and by only considering initial and final entries of $\rho_D$ that are diagonal, i.e. $\oph{i}{i}$. In that case all entries of the kernel will be purely real-valued  and the resulting integrals have particularly simple solutions (see Sec.~\ref{Sec:Integrals}). Due to charge conservation the kernel only connects states differing by a single electron. This kernel is equivalent to the kernel in the \verb!Pauli! approach. 
\\
\\
The diagram and resulting integral expression can also be derived using the operator definitions \ref{Eq:G1}. This is done in great detail in Ref.~\cite{leijnse2008kinetic}.
\\
\\
In QmeQ the function \verb!ApproachRTD.generate_row_1st_order_kernel()! generates a row of $W_{dd}^{(1)}$.

\subsection{$W^{(2)}_{dd}$}

In second order ($\propto\Gamma^2$) there are two diagrammatic contributions arising from contracting different $\mathcal{J}$s using Wick's theorem. Using the operators in \ref{Eq:G2} and tracing over the leads gives the contributions~\cite{gergs2015charge, gergs2017transport, gergs2018spin}

\begin{equation}
	\begin{split}
    -i\iint d\omega_1 d\omega_2\sum_{1234}\frac{t_1t_2t_3t_4}{(2\pi)^2}\left (\G_{\bar{4}}^+\frac{\delta_{\bar{\eta}_1\eta_4}\delta_{\bar{\eta}_2\eta_3}}{i0^++\eta_1\omega_1-L_D} \G_{\bar{3}}^{+}-\G_{\bar{4}}^+\frac{\delta_{\eta_1\bar{\eta}_3}\delta_{\bar{\eta}_2\eta_4}}{i0^++\eta_2\omega_2-L_D}\G_{\bar{3}}^{+} \right ) & \times \\
      \frac{\bar{q}_2\gamma_2^{\bar{q}_2}}{i0^++\eta_1\omega_1 + \eta_2\omega_2-L_D}\G_2^{q_2}&\frac{\bar{q}_1\gamma_1^{\bar{q}_1}}{i0^++\eta_1\omega_1-L_D}\G_1^{q_1}.
     \end{split}
    \label{Eq:W2Gergs}
\end{equation}
\\
However, for this kernel Ref.~\cite{leijnse2008kinetic} gives an explicit expression for the matrix elements using the operators in Eq. \ref{Eq:G1}, and that expression is used as a basis for the kernel in QmeQ. The expression for the matrix element connecting $\oph{a_0}{a_0}$ to $\oph{a_4}{a_4}$ then reads

\begin{align}
	\begin{split}
    \left [ W^{(2)}_{dd} \right ]^{a_0a_0}_{a_4a_4} = -i \sum_{p1,p2,p3,p4}\sum_{r_1,r_2}\sum_{\eta_1\eta_2}\sum_{a_{3\pm}a_{2\pm}a_{1\pm}} \Bigg \{ &
     \left  (G^{p_4}_{\bar{\eta_1}r_1}\right)^{a_3}_{a_4} \left (G^{p_3}_{\bar{\eta_2}r_2}\right)^{a_2}_{a_3}  \left (G^{p_2}_{\eta_1r_1}\right)^{a_1}_{a_2}  \left (G^{p_1}_{\eta_1r_1}\right)^{a_0}_{a_1} \times \\ 
     &\iint d\omega_1 d\omega_2 \frac{p_1p_2f(p_2[\omega_2-\eta_2\mu_2]/T_2)f(p_1[\omega_1-\eta_1\mu_1]/T_1)}{(i0^++\omega_1-E_{a_3})(i0^++\omega_1 + \omega_2-E_{a_2})(i0^++\omega_1-E_{a_1})}\\
      - \left (G^{p_4}_{\bar{\eta_2}r_2}\right)^{a_3}_{a_4} &\left (G^{p_3}_{\bar{\eta_1}r_1}\right)^{a_2}_{a_3}  \left (G^{p_2}_{\eta_2r_2}\right)^{a_1}_{a_2}  \left (G^{p_1}_{\eta_1r_1}\right)^{a_0}_{a_1} \times \\
        &\iint d\omega_1d\omega_2 \frac{p_1p_2f(p_2[\omega_2-\eta_2\mu_2]/T_2)f(p_1[\omega_1-\eta_1\mu_1]/T_1)}{(i0^++\omega_2-E_{a_3})(i0^++\omega_1 + \omega_2-E_{a_2})(i0^++\omega_1-E_{a_1})}  \Bigg \}.
	\end{split}
    \label{Eq:W2Leijnse}
\end{align}
\\
Here $a_n$ is shorthand notation for a pair of many-body states $\ket{a_{n+}}, \ket{a_{n-}}$ and $E_{a_n}$ is the energy difference $E_{a_{n+}}-E_{a_{n-}}$. In Eq. \ref{Eq:W2Leijnse} the tunneling amplitudes are included in the $G$ operators such that their matrix elements reads

\begin{equation}
    \left ( G^{p}_{\eta r}\right )^{a_1}_{a_2} = p^{1+ N_{a_{2+}}-N_{a_{2-}}} T^{a_{2_p}a_{1_p}}_{r(p\cdot\eta)}\delta_{a_{2_{\bar{p}}}a_{1_{\bar{p}}}}
\end{equation}
\\
where $N_{a_{n_\pm}}$ is the charge of state $\ket{a_{n_\pm}}$ and 

\begin{align}
    &T^{aa'}_{r(+)} = \sqrt{\nu_r}\sum_l t_{lr} \bra{a}d_l^\dagger \ket{a'},\\
    \label{Eq:manyBodyT}
    T^{aa'}_{r(-)} &= \sqrt{\nu_r}\sum_l t^*_{lr} \bra{a}d_l \ket{a'} = \left (T^{aa'}_{r(+)}\right ) ^*.
\end{align}
\\
The solutions to the integrals are given in Sec.~\ref{Sec:Integrals}. Since solving these integrals are the most computationally heavy task in the calculations they are only solved if $\Gamma/T>10^{-10}$, otherwise they are ignored.
\\
\\
In QmeQ certain symmetries between sub-contributions of the matrix elements are used to reduce the number of variables needed to be summed over. The symmetries make it possible to add each contribution $w^{a_0}_{a_4}(\theta)$ obtained by solving in Eq. \ref{Eq:W2Leijnse} for a set of $\theta=\{p_1,p_2,p_3,p_4,\eta_1,\eta_2\, a_{3_\pm}, a_{2_\pm}, a_{1_\pm} \}$ also to other elements $\left [ W^{(2)}_{dd} \right ]^{a_ia_i}_{a_ja_j}$ of the kernel. For each term in the sum in Eq. \ref{Eq:W2Leijnse} the result is added as 

\begin{align}
    \left [ W^{(2)}_{dd} \right ]^{a_0a_0}_{a_4a_4}\ += 2w^{a_0}_{a_4}(\theta)\label{Eq:W2symmetries1}\\
    \left [ W^{(2)}_{dd} \right ]^{a_0a_0}_{a_{3_+}a_{3_+}} -= 2w^{a_0}_{a_4}(\theta)\\
    \left [ W^{(2)}_{dd} \right ]^{a_{1_+}a_{1_+}}_{a_4a_4} += 2w^{a_0}_{a_4}(\theta)\\
    \left [ W^{(2)}_{dd} \right ]^{a_{1_+}a_{1_+}}_{a_{3_+}a_{3_+}} -= 2w^{a_0}_{a_4}(\theta)\label{Eq:W2symmetries4}
\end{align}
\\
where the second row stems from setting $p_4=-p_4$, the third from $p_1=-p_1$ and the fourth from $p_1=-p_1, p_4=-p_4$. The factor 2 comes from $w^{a_0}_{a_4}(\theta)\big|_{p, \eta}=\left (w^{a_0}_{a_4}(\theta)\right)^*\big|_{\bar{p}, \bar{\eta}}$. As a result we can set $p_1=p_4=\eta_1=1$ when evaluating the matrix elements according to Eq. \ref{Eq:W2Leijnse}. 
\\
\\
Adding $w^{a_0}_{a_4}(\theta)$ at the correct places in $W^{(2)}_{dd}$ is done in \verb!KernelHandlerRTD.add_element_2nd_order()!.
\\
\\
The complete matrix $W_{dd}^{(2)}$ is build by calling \verb!ApproachRTD.generate_col_diag_kern_2nd_order()! for each initial state. 
\\
\\
\subsection{$W_{dn/nd}^{(1)}$}

These (sub)kernels stem from the same diagram as $W^{(1)}_{dd}$. However, when off-diagonal terms in $\rho_D$ are included the matrix elements are no longer guaranteed to be real and the solutions to the integrals become complex. See Section~\ref{Sec:Integrals}.
\subsection{Kernels for the energy current}

As mentioned previously in Sec.~\ref{Sec:Observables}, correction terms are needed when evaluating the energy current in a reservoir using local conservation laws. 
\\
\\
The starting point for deriving the correction terms is the reservoir observable we are interested in: $\left \langle -\frac{dH_r}{dt} \right \rangle= \langle-i[H, H_r]\rangle = \langle-i[H_{T,r}, H_r]\rangle$. Rewriting the commutator using  Eq. \ref{Eq:Hamiltonian} yields

\begin{equation}
    -[H_{T,r}, H_r] = [H_{T,r}, H_D] + [H_{T,r}, \sum_{r\ne r'}H_{T,r'}] - [H_{T,r}, H].
\end{equation}
\\
The last term corresponds to $\frac{d}{dt}H_{T,r}$, which is $0$ in our case. The first term can be treated exactly as described previously in Sec.~\ref{Sec:Observables} for the charge current, which gives

\begin{equation}
    \langle J_D\rangle = \frac{1}{2}\Tru{D}L_D^+ W_r\rho_D, \ \ \ L_D^+=[H_D,  \bullet]_+ .
\end{equation}
\\
We will now focus on the correction term $[H_{T,r}, \sum_{r\ne r'}H_{T,r'}]$. The expectation value of this terms can be written as (assuming implicit summation over $r'$ such that $r'\ne r$)

\begin{equation}
    \Tr\left ( i[H_{T,r}, H_{T,r'}]\rho_D \right ) = \frac{i}{2}\Tr\left ( L_{T,r}^+L_{T,r'} \rho_D \right ),
\end{equation}
\\
which can be seen by simply writing out both terms and using the cyclic property of the trace. This allows us to rewrite the expectation value as 

\begin{equation}
	\begin{split}
   \frac{i}{2}\Tr  \left [ L_{T,r}^+L_{T,r'} \frac{i}{z-L_D-L_R-L_T}\rho(t_0) \right ]  &= \frac{i}{2}\Tr \left [ L_{T,r}^+L_{T,r'} \frac{1}{z-L_D-L_R}L_T\frac{1}{z-L_D-L_R}L_T\frac{i}{z-L_D-L_R-L_T}\rho(t_0) \right ] \\ 
   =\frac{i}{2}\Tr  &\left [ L_{T,r}^+L_{T,r'} \frac{1}{z-L_D-L_R}L_T\frac{1}{z-L_D-L_R}L_T\rho(z) \right ] ,
	   \end{split}
\end{equation}
\\
which has its first contribution at order $\mathcal{O}(\Gamma^2)$. Following Ref.~\cite{gergs2017transport} we replace $r$ with $r_1$ ($r_1$ is not summed over) and write

\begin{equation}
    \frac{i}{2}\Tr \sum_{\underset{r_2\ne r_1}{1234,r_2}}t_1t_2t_3t_4\eta_1\eta_2\eta_3\eta_4 \sum_{q_1, q_2, q_3, q_4}\G^{q_1}_1\J^{q_1}_{\bar{1}}\G^{\bar{q}_2}_2\J^{q_2}_{\bar{2}}\frac{1}{z-L_D-L_R}\G^{\bar{q}_3}_3\J^{q_3}_{\bar{3}}\frac{1}{z-L_D-L_R}\G^{\bar{q}_4}_1\J^{q_4}_{\bar{4}}\rho(z).
\end{equation}
\\
Now we use $\bral{\mathbb{1_R}}\J^+=0$ to fix $q_1=-$. Since $r_1\ne r_2$ and $\Tru{R} = \sum_r \Tru{r}$ we also see that $q_2=-$. For the first four superoperators we then have 

\begin{equation}
    \G^-_1\J^-_{\bar{1}}\G^+_2\J^-_{\bar{2}} = \G^-_1\G^+_2\J^-_{\bar{1}}\J^-_{\bar{2}} = -\G^+_2\G^-_1\J^-_{\bar{1}}\J^-_{\bar{2}} + \delta^D_{1\bar{2}}\J^-_{\bar{1}}\J^-_{\bar{2}} = \delta^D_{1\bar{2}}\J^-_{\bar{1}}\J^-_{\bar{2}},
\end{equation}
\\
where $[\G^-_1, \G^+_2]_+=\delta^D_{1\bar{2}}$ and $\bral{\mathbb{1}^D}\G^+=0$ have been used. $\delta^D_{1\bar{2}}$ only acts on the QD degrees of freedom $\delta^D_{1\bar{2}}=\delta_{l_1l_2}\delta_{\eta_1\bar{\eta}_2}$ keeping $r_1\ne r_2$. This leaves us with
\\
\begin{equation}
    \frac{i}{2}\Tr \sum_{\underset{r_2\ne r_1}{134,r_2 }}t_1t_2t_3t_4\eta_1\eta_2\eta_3\eta_4 \delta^D_{1\bar{2}}\sum_{q_3, q_4}\left( \J^-_{\bar{1}}\J^-_{\bar{2}} \frac{1}{z-L_D-L_R}\G^{\bar{q}_3}_3\J^{q_3}_{\bar{3}}\frac{1}{z-L_D-L_R}\G^{\bar{q}_4}_4\J^{q_4}_{\bar{4}} \right )\rho.
\end{equation}
\\
To continue we identify the following
\begin{itemize}
	\item $\bral{\mathbb{1}^D}L_D=0$ removes $L_D$ from the first propagator
	\item The left-most QD operator is then $\G^{\bar{q}_3}_3$ which sets $\bar{q}_3=-$ since $\bral{\mathbb{1}^D}\G^+=0$
	\item $\delta^D_{1\bar{2}}$ sets $\eta_1\eta_2=-\eta_1^2=-1$ and $t_1t_2=t_{\omega_1, r_1, l_1, \eta_1}t_{\omega_2, r_2, l_1, \bar{\eta_1}}$
 \end{itemize}

Next we pull all $\J$s through to the right using $L_R\J_1^q=\J_1^{q}(L_R+\eta_1\omega_1)$ and $\bral{\mathbb{1}_R}L_R=0$
 
 \begin{equation}
     -\frac{i}{2}\Tru{D} \sum_{\underset{r_2\ne r_1}{134,r_2}}t_{\omega_1, r_1, l_1,\eta_1}t_{\omega_2, r_2, l_1, \bar{\eta_1}}t_3t_4\eta_3\eta_4\delta^D_{1\bar{2}}\sum_{ q_4}\frac{1}{z-\eta_1\omega_1- \eta_2\omega_2}\G^-_3\frac{1}{z-\eta_1\omega_1-\eta_2\omega_2-\eta_3\omega_3-L_D}\G^{\bar{q_4}}_4\Tru{R}\J^-_{\bar{1}}\J^-_{\bar{2}}\J^+_{\bar{3}}\J^{q_4}_{\bar{4}}\rho_R\rho_D.
\end{equation}
 \\
 Now we use the fact that we get two contributions from Wick's theorem when tracing out the reservoirs: $\langle\J^-_{\bar{1}}\J^{q_4}_{\bar{4}}\rangle_R\langle\J^-_{\bar{2}}\J^+_{\bar{3}}\rangle_R$ and $\langle\J^-_{\bar{1}}\J^+_{\bar{3}}\rangle_R\langle\J^-_{\bar{2}}\J^{q_4}_{\bar{4}}\rangle_R$, which provide the two kernels $W_{E,1}$ and $W_{E,2}$, respectively.
 %------------------------
 \subsection{$W_{E,1}$}
 
 Focusing on the first contribution we get $\langle\J^-_{\bar{2}}\J^+_{\bar{3}}\rangle_R=\delta_{\bar{2}3}^R$ and $\langle\J^-_{\bar{1}}\J^{q_4}_{\bar{4}}\rangle_R=\delta_{\bar{1}4}^R\tanh\left ( \frac{\eta_1(\omega_1-\mu_1)}{2T_1}\right )$ with $\delta_{\bar{1}2}^R=\delta_{r_1,r_2}\delta_{\omega_1, \omega_2}\delta_{\bar{\eta}_1,\eta_2}$ and thus
 
\begin{equation}
    -\frac{i}{2}\Tru{D} \sum_{134,l,r_2 (r_2\ne r_1)}t_{\omega_1, r_1, l,\eta_1}t_{\omega_2, r_2, l,\bar{\eta_1}}t_3t_4 \delta^D_{1\bar{2}} \delta_{\bar{2}3}^R\delta_{\bar{1}4}^R \sum_{ q_4}\frac{1}{z-\eta_1\omega_1- \eta_2\omega_2}\G^-_3\frac{\tanh\left ( \frac{\eta_1(\omega_1-\mu_1)}{2T_1} \right )}{z-\eta_1\omega_1-L_D}\G^{\bar{q_4}}_4\rho_D.
\end{equation}
\\
Using the Sohotski-Plemelj theorem we can perform the integral over $\omega_2$ by closing the contour over the upper/lower half-plane for $\eta_2=\pm$. This leaves us with 

\begin{equation}
     -\frac{1}{2}\Tru{D} \int d\omega_1\sum_{12,l,r_2 (r_2\ne r_1)}\pi\delta_{\bar{\eta_1}\eta_2}t_{\omega_1, r_1, l\eta_1}t_{\omega_2, r_2, l,\bar{\eta_1}}t_{\bar{2}}t_{\bar{1}}\G^-_{\bar{2}}\frac{\tanh\left ( \frac{\eta_1(\omega_1-\mu_1)}{2T_1} \right )}{z-\eta_1\omega_1-L_D}\G^{+}_{\bar{1}}\rho_D = -\frac{1}{2}\Tru{D} W_{E,1}^r\rho_D.
\end{equation}
\\
Here one can immediately notice the similarities between this expression and Eq. \ref{Eq:Wdd1}. The differences being that this contribution is of order $t^4$ (i.e. $\Gamma^2)$, and that this expression is not multiplied with a factor $i$, which the expression in Eq. \ref{Eq:Wdd1} is. We thus need the real part of the (first order) integral in order for our matrix elements of the kernel to be real (we're still connecting diagonal to diagonal parts of $\rho_D$). Regarding the tunneling amplitudes, $t_{\bar{2}}t_{\bar{1}}$ can be combined with the $\G$s when evaluating matrix elements and thus be expressed as \textit{many-body} tunneling amplitudes (see Eq. \ref{Eq:manyBodyT}), whereas $t_{\omega_1, r_1, l_1}t_{\omega_2, r_2, l_1}$ are \textit{single particle} tunneling amplitudes connecting the same single particle state to two reservoirs.
\\
\\
The similarities with $W_{dd}^{(1)}$ are used in QmeQ when building $W_{E,1}$ and $W_{E,2}$. The only differences between $W_{E,1/2}$ and $W_{dd}^{(1)}$ are that the matrix elements are based on the real part of the integral and that a pre-factor stemming from $t_{\omega_1, r_1, l_1}t_{\omega_2, r_2, l_1}$ is included in each matrix element.
\\
\\ 
The function \verb!ApproachRTD.generate_row_1st_energy_kernel()! calculates a row in $W_{E,1}$. The implementation assumes that the product of $t$s is purely real (the individual $t$s can still be imaginary), and will print an error message if it's not.
 %--------------------------------------
 \subsection{$W_{E,2}$}

Using the same approach as for $W_{E,1}$ we can derive the second contribution arising from contracting $\J_1$ with $\J_3$ and $\J_2$ with $\J_4$. By going through the same steps as above we find the contribution

\begin{equation}
     \frac{1}{2}\Tru{D} \int d\omega_2\sum_{\underset{r_2\ne r_1}{12,l,r_2}}\pi\delta_{\bar{\eta_1}\eta_2}t_{\omega_1, r_1, l}t_{\omega_2, r_2, l}t_{\bar{2}}t_{\bar{1}}\G^-_{\bar{1}}\frac{\tanh\left ( \frac{\eta_2(\omega_2-\mu_2)}{2T_2} \right )}{z-\eta_2\omega_2-L_D}\G^{+}_{\bar{2}}\rho_D = \frac{1}{2}\Tru{D} W_{E,2}^r\rho_D.
\end{equation}
\\
The function \verb!ApproachRTD.generate_row_2nd_energy_kernel()! calculates a row in $W_{E,2}$. The products of $t$s are assumed to be real, just as for the first kernel.
%--------------------------------------
\subsection{Energy and heat current}

The total energy current out of lead $r$ is given by 

\begin{equation}
    \langle J_r \rangle = \Tru{D} L_D W_r \rho_D - \frac{1}{2}\Tru{D}W_{E,1}^r\rho_D + \frac{1}{2}\Tru{D}W_{E,2}^r\rho_D,
\end{equation}
\\
and the heat current is given by the first law of thermodynamics

\begin{equation}
    \langle Q_r \rangle = \langle J_r \rangle - \mu_r\langle I_r \rangle
\end{equation}
%-------------------------------------------------- Section:Integrals
\section{Integrals}
\label{Sec:Integrals}

There are a number of complex-valued integrals that need to be solved when setting up the various kernels used in the RTD approach. The solutions to these integrals are based on Cauchy's residue theorem. Some integrals also make use of the Sokhotski-Plemelj-theorem, which states that

\begin{equation}
    \frac{1}{x+E+i0^+}f(x) = \mathcal{P}\frac{1}{x+E}f(x)-i\pi\delta(x+E)f(x),
\end{equation}
 \\
where $\mathcal{P}$ denotes the principal value.
%-------------------------------------------
\subsection{First order kernels}

The matrix elements of the first order kernels $W_{dd}^{(1)}$, $W_{nd}^{(1)}$, $W_{dn}^{(1)}$ as well as for the correction terms for the energy current $W_{E,1}$ and $W_{E,1}$, are based on the same integral. It reads

\begin{equation}
    \frac{1}{2}\int d\omega_1 \frac{\tanh\left (\frac{\eta_1(\omega_1-\mu_1)}{2T_1}\right )}{\eta_1\omega_1 + E + i0^+},
\end{equation}
\\
where $E$ denotes an energy difference stemming from $L_D$ in Eq. \ref{Eq:Wdd1}. In order to solve the integral we introduce a bandwidth $D$ (which should be the, by far, largest energy in the problem) and change our integration limits from $\pm\infty$ to $\pm D$. Then we perform a change of variables $x = \eta_1(\omega_1-\mu_1)$ and use the residue theorem by defining a closed contour as a semi-circle on the upper half of the complex plane. This results in 

\begin{equation}
    \frac{1}{2}\int_D^Ddx \frac{\tanh\left (\frac{x}{2T_1}\right )}{x + \eta_1\mu_1 + E + i0^+} = -\frac{1}{2}\int_0^\pi d\phi \frac{z\tanh\left (\frac{z}{2T_1}\right )}{z + \eta_1\mu_1+ E + i0^+}\Bigg |_{z=De^{i\phi}} + 2\pi T_1 i\sum_{n=0}^{k_D} \frac{1}{z+\eta_1\mu_1 + E+i0^+}\Bigg |_{z=i\pi(1+2k)T_1},
\end{equation}
\\
since the residues and (Matsubara) poles of $\tanh\left ( \frac{z}{2T_1} \right )$ are $2T_1$ and $i\pi T_1(1+2n)$, respectively. $k_D$ is the integer part of $\frac{D}{2\pi T_1}-\frac{1}{2}$. The integral over the arc away from the real axis will vanish, leaving us with

\begin{equation}
   \frac{1}{2}\int_D^Ddx \frac{\tanh\left (\frac{x}{2T_1}\right )}{x + \eta_1\mu_1 + E + i0^+}= \sum_{n=0}^{k_D} \frac{1}{\left ( n+\frac{1}{2}\right ) - i\frac{E+\eta_1\mu_1}{2\pi T_1}}.
\end{equation}
\\
Next we make use of the fact that $D\rightarrow \infty$ in the wide-band limit to add and subtract the Euler constant $\gamma_E=\lim_{k\rightarrow \infty}\sum_{l=1}^{k+1}1/l-\ln l$.

\begin{equation}
   \frac{1}{2}\int_D^Ddx \frac{\tanh\left (\frac{x}{2T_1}\right )}{x + \eta_1\mu_1 + E + i0^+}= \sum_{n=0}^{k_D} \left ( \frac{1}{\left ( n+\frac{1}{2}\right ) - i\frac{E+\eta_1\mu_1}{2\pi T_1}}-\frac{1}{n+1}\right ) + \gamma_E +\ln k_D \approx -\psi\left ( \frac{1}{2}-i\frac{E+\eta_1\mu_1}{2\pi T_1}\right )  + \ln \frac{D}{2\pi T_1},
\end{equation}
\\
where $\psi$ is the complex-valued digamma function. The real part of of the integral is denoted by $\phi(-x_1) = -\Re \psi\left ( \frac{1}{2}-i\frac{x_1}{2\pi}\right )  + \ln \frac{D}{2\pi T}$ in QmeQ. Note that $\phi(x_1)$ is symmetric for real $x$ and we have chosen to include the extra sign so that $\phi(x)$ agrees with the definition in Ref.~\cite{leijnse2008kinetic} (which contains a detailed solution of the integral). $\phi(x)$ is used when setting up $W_{nd}^{(1)}$, $W_{dn}^{(1)}$, $W_{E,1}$ and $W_{E,1}$. For $W_{dd}^{(1)}$ we instead need the imaginary part of the integral, which we can get directly using the Sokhotski-Plemelj-theorem and $-i\frac{\pi}{2}\tanh\left ( \frac{\eta_1(\omega_1-\mu_1)}{2T_1}\right )=-i\frac{\pi}{2}+i\pi f\left(\frac{\eta_1(\omega_1-\mu_1)}{T_1}\right )$. This kernel thus only requires evaluations of the Fermi function $f(x)$.
\\
\\
When evaluating the second order integrals we'll use the notation $\tilde{\phi}(z_1) = -\psi(z_1) + \ln \frac{D}{2\pi T_1}$
%-----------------------------------------------
\subsection{Second order kernel}

In the second order kernel $W_{dd}^{(2)}$ the are two types of integrals that need to be solved, the "direct" integral
\\
\begin{equation}
    I_D = \iint d\omega_1 d\omega_2 \frac{p_1p_2f(p_2[\omega_2-\eta_2\mu_2]/T_2)f(p_1[\omega_1-\eta_1\mu_1]/T_1)}{(i0^++\omega_1-E_{a_3})(i0^++\omega_1 + \omega_2-E_{a_2})(i0^++\omega_1-E_{a_1})},
\end{equation}
\\
and the "exchange" integral 

\begin{equation}
    I_X = \iint d\omega_1d\omega_2 \frac{p_1p_2f(p_2[\omega_2-\eta_2\mu_2]/T_2)f(p_1[\omega_1-\eta_1\mu_1]/T_1)}{(i0^++\omega_2-E_{a_3})(i0^++\omega_1 + \omega_2-E_{a_2})(i0^++\omega_1-E_{a_1})}.
\end{equation}
\\
We divide the solutions of these integrals into two cases: $T_1=T_2$ and $T_1\ne T_2$.
%----------------------------------------------------
\subsubsection{Direct integral :$T_1=T_2$}

For the special case of equal temperatures the imaginary part of the integral can be solved for analytically. This is done in Ref.~\cite{leijnse2008kinetic}, which yields

\begin{equation}
\begin{split}
    I_D = \frac{p_1p_2}{T}\frac{F\left [(E_{a_2}-\eta_1\mu_1-\eta_2\mu_2)/T, (E_{a_3}-\eta_1\mu_1)/T\right]- F\left[(E_{a_2}-\eta_1\mu_1-\eta_2\mu_2)/T, (E_{a_1}-\eta_1\mu_1)/T\right]}{(E_{a_3}-E_{a_1})/T} \\
    +\frac{p_1(1-p_2)}{T}\frac{\tilde{F}\left [(E_{a_3}-\eta_1\mu_1)/T\right ] - \tilde{F}\left [ (E_{a_1}-\mu_1\eta_1)/T \right] }{(E_{a_3}-E_{a_1})/T},
\end{split}
\end{equation}
\\
where $\tilde{F}(x)=\frac{\pi}{2}\phi(x)$ and 

\begin{equation}
    F(\lambda', \lambda) = \pi\cdot
    \begin{cases}
         \phi(\lambda'-\lambda)f(\lambda) + b(\lambda')[\phi(\lambda'-\lambda)-\phi(-\lambda)] , \ \ \lambda'\ne 0\\
         \phi(-\lambda)f(\lambda)-\frac{d}{d\lambda}\phi(\lambda) , \ \ \lambda'\rightarrow 0
    \end{cases} .
\end{equation}
\\
$f(\lambda)$ and $b(\lambda)$ are the Fermi and Bose distributions, respectively. The case when $E_{a_3}=E_{a_1}$ (numerically $|E_{a_3}/T-E_{a_1}/T|<10^{-10}$) is handled using L'Hospital's rule that states that if $\lim_{x\rightarrow c}f(x) = \lim_{x\rightarrow c}g(x) = 0$ and if $f$ and $g$ are differentiable then $\lim_{x\rightarrow c}\frac{f(x)}{g(x)} = \lim_{x\rightarrow c}\frac{f'(x)}{g'(x)}$.
%-----------------------------------------------------------
\subsubsection{Exchange integral: $T_1=T_2$}

The exchange integral is given by (again from Ref.~\cite{leijnse2008kinetic})

\begin{equation}
\begin{split}
    I_X = p_2p_1\frac{F\left [ (E_{a_2}-\eta_1\mu_1-\eta_2\mu_2)/T , (E_{a_1}-\eta_1\mu_1)/T \right ] - F\left [ (E_{a_3} + E_{a_1}-\eta_1\mu_1-\eta_2\mu_2)/T , (E_{a_1}-\eta_1\mu_1)/T \right ]}{(E_{a_2}-E_{a_3}-E_{a_1})/T} \\
    + p_2p_1\frac{F\left [ (E_{a_2}-\eta_1\mu_1-\eta_2\mu_2)/T , (E_{a_3}-\eta_2\mu_2)/T \right ] - F\left [ (E_{a_3} + E_{a_1}-\eta_1\mu_1-\eta_2\mu_2)/T , (E_{a_3}-\eta_2\mu_2)/T \right ]}{(E_{a_2}-E_{a_3}-E_{a_1})/T},
\end{split}
\end{equation}
\\
where the case $E_{a_2} = E_{a_3} + E_{a_1}$ (numerically $|E_{a_2}/T - E_{a_3}/T - E_{a_1}/T|<10^{-10}$) is handled using L'Hospital's rule.
%-----------------------------------------------------------
\subsubsection{Ozaki expansion}

When the two temperatures entering the integrals are not equal we can no longer obtain a nice analytical expression for the solution. One can usually solve one of the two integrals using the residue theorem. However, when using the residue theorem a second time it's not always possible to convert the sum to a digamma function. This leaves us in the worst case with a large sum (over the complex poles of $\tanh$) where each term in the sum requires evaluating the digamma function, which is a computationally heavy task. 
\\
\\
A better approach is to use the Ozaki expansion of the Fermi function (or $\tanh$) when performing the second integral. The Ozaki expansion is introduced in Ref.~\cite{ozaki2007continued}, but here we'll use the notation of Ref.~\cite{karrasch2010finite}. The expansion approximates $\tanh$ as

\begin{equation}
    -\frac{\tanh(x/2)}{2} = f(x) - \frac{1}{2} \approx -\sum_{\alpha>0}\left( \frac{R_\alpha}{x-i/b_\alpha}+\frac{R_\alpha}{x+i/b_\alpha}\right ).
\end{equation}
\\
Here $b_\alpha$ is an eigenvalue of the matrix $B$, $B\ket{B_\alpha}=b_\alpha\ket{B_\alpha}$, which is a band matrix with all elements zero except for the elements next to the diagonal. Those are given by

\begin{equation}
    B_{n,n+1} = B_{n+1,n}=\frac{1}{2\sqrt{(2n-1)(2n+1)}}.
\end{equation}
\\
The residues are approximated as $R_\alpha=\frac{|\langle 1\ket{b_\alpha}|^2}{4b_\alpha^2}$, where $\langle 1\ket{b_\alpha}$ is the first element in an eigenvector of $B$. The eigenvalues come in pairs ($\pm|b_\alpha|$), but only positive eigenvalues (and associated eigenvectors) should be kept, which is indicated by $\alpha>0$ in the expansion. 
\\
\\
The largest value of $b_\alpha$ is determined by the size of the matrix $B$, and a larger matrix naturally improves the approximation. In QmeQ the user provides a bandwidth $D$, and not the size of the matrix. The size of the matrix is instead set such that the largest value of $b_\alpha$ is approximately equal to $D$, and the user should thus only have to increase $D$ in order to increase the accuracy of the calculations. The matrix size $N$ is calculated based on the empirical expression for the maximum eigenvalue $b_{max}$

\begin{equation}
    b_{max} = \frac{D}{\text{min}(\ \texttt{leads.tlst}\ )}= 2.55N^{2} + 1.27N + 0.52.
\end{equation}
\\
This can be contrasted with the Matsubara expansion where the bandwidth scales linearly with the number of poles.
\\
\\
Diagonalizing the matrix $B$ is a quite expensive operation, and it is thus only done when needed. It is for example not calculated if all temperatures are equal. For unequal temperatures it is calculated before solving any integral for the first time, and then only re-calculated when $T$ or $D$ are changed such that $\frac{D}{T}$ is larger than before.
%-------------------------------------------------------------
\subsubsection{Direct integral: $T_1\ne T_2$}

For the case of non-equal temperatures we'll use the result presented in the SI of Ref.~\cite{gergs2018spin}. Those results are based on the diagram expression in Eq. \ref{Eq:W2Gergs}, which uses $\sum_{q_1q_2}q_1q_2\gamma^{q_1}_1\gamma^{q_2}_2$ instead of $f(x_1)f(x_2)$. However, using $f(x)=\frac{1}{2}-\tanh\left(\frac{x}{2}\right)$ we can write

\begin{equation}
    \iint f(x_1)f(x_2) = \iint \left(\frac{1}{2}-\frac{1}{2}\tanh\left(\frac{x_1}{2}\right)\right)\left(\frac{1}{2}-\frac{1}{2}\tanh\left(\frac{x_2}{2}\right)\right),
\end{equation}
\\
allowing us to use the results in Ref.~\cite{gergs2018spin}. Furthermore, it turns out that two contributions vanish, meaning we only have to evaluate $\frac{1}{4}\iint \tanh\left(\frac{x_1}{2}\right)\tanh\left(\frac{x_2}{2}\right) - \frac{1}{4}\iint\tanh\left(\frac{x_1}{2}\right)$. We thus want to solve 

\begin{equation}
    I_D = \frac{1}{4}I_D^--\frac{1}{4}I_D^+ = -\frac{1}{4}\sum_{q_2}q_2 \int_{-D}^D\int_{-D}^Dd\omega_1d\omega_2\frac{1}{\omega_1+E_3+ i0^+}\cdot\frac{\gamma^{q_2}_2\gamma^{-}_1}{\omega_1 + \omega_2 + E_2 + i0^+}\cdot\frac{1}{\omega_1+E_1+i0^+},
\end{equation}
\\
where $E_n= -E_{a_n}$. For the first contribution ($q_2=+$) we have

\begin{equation}
\begin{split}
   I_D^+ =\int_{-D}^D\int_{-D}^Dd\omega_1&d\omega_2  \frac{1}{\omega_1+E_3+ i0^+}\cdot\frac{\gamma^{-}_1}{\omega_1 + \omega_2 + E_2 + i0^+}\cdot\frac{1}{\omega_1+E_1+i0^+} = \\ 
   - \frac{1}{E_3-E_1}\int_{-D}^D&\int_{-D}^Dd\omega_1d\omega_2\frac{\gamma^-_1}{\omega_1 + \omega_2 + E_2 + i0^+}\left( \frac{1}{\omega_1+E_3} -\frac{1}{\omega_1 + E_1} \right ) = \\ 
   -\frac{i\pi}{E_3-E_1}&\int_{-D}^D d\omega_1\tanh\left( \frac{\omega_1-\eta_1\omega_1}{2T_1}\right )\left ( \frac{1}{\omega_1+E_3}- \frac{1}{\omega_1+E_1}\right ) = \\
    -&2\pi i \frac{\psi\left(\frac{1}{2}-i\frac{E_3+\eta_1\mu_1}{2\pi T_1}\right)-\psi\left(\frac{1}{2}-i\frac{E_1+\eta_1\mu_1}{2\pi T_1}\right)}{E_3-E_1}.
    \end{split}
\end{equation}
\\
When $E_3=E_1$ (numerically $|E_3/T_1-E_1/T_1|<10^{-10}$) we use L'Hospital's rule to get $I_D^+=-\frac{1}{T_1}\psi'\left( \frac{1}{2}-i\frac{E_3+\eta_1\mu_1}{2\pi T_1} \right)$. For the second contribution ($q_2=-$) we can use the normal Matsubara expansion to obtain

\begin{equation}
\begin{split}
    I_D^- = \int_{-D}^D\int_{-D}^Dd\omega_1d\omega_2 \frac{1}{\omega_1+E_3+ i0^+}&\cdot\frac{\gamma^{-}_1\gamma^{-}_2}{\omega_1 + \omega_2 + E_2 + i0^+}\cdot\frac{1}{\omega_1+E_1+i0^+} = \\ 
    2\int_{-D}^Dd\omega_1\tilde{\phi}\left( \frac{1}{2}-i\frac{\omega_1+E_2+\eta_2\mu_2}{2\pi T_2}\right)&\frac{1}{\omega_1+E_3}\cdot\frac{1}{\omega_1 + E_1}\tanh\left ( \frac{\omega_1-\eta_1\mu_1}{2T_1}\right ) = \\ 
    -i8 T_1\pi \sum_{n=0}^{k_D}\tilde{\phi}\left ( \frac{1}{2} + \frac{T_1}{T_2}\left(n + \frac{1}{2} \right )-i\frac{E_2 + \eta_1\mu_1 + \eta_2\mu_2}{2\pi T_2} \right )&\frac{1}{2\pi T_1(n + 1/2)-i(E_3+\eta_1\mu_1)}\cdot\frac{1}{2\pi T_1(n+1/2)-i(E_1+\eta_1\mu_1)}.
\end{split}
\end{equation}
\\
Using the Ozaki expansion this can instead be written as 

\begin{equation}
    I_D^- = -i8T_1\pi \sum_{\alpha>0}\tilde{\phi}\left ( \frac{1}{2}+\frac{b_\alpha T_1}{2\pi T_2} -i\frac{E_2+\eta_1\mu_1+\eta_2\mu_2}{2\pi T_2} \right )\frac{R_\alpha}{T_1b_\alpha-i(E_3+\eta_1\mu_1)}\cdot\frac{1}{T_1b_\alpha-i(E_1+\eta_1\mu_1)}.
\end{equation}
\\
When taking into account the Keldysh signs $p$ to make the integrals compatible with the diagram in Eq. \ref{Eq:W2Leijnse} the result is $I_D = \frac{1}{4}(p_2p_1I_D^--p_1I_D^+)$.
%-----------------------------------------------------
\subsubsection{Exchange integral: $T_1\ne T_2$}

For the exchange integral we again rewrite the product of Fermi functions using $\tanh$. However, this time only the term $\frac{1}{4}\iint \tanh\left(\frac{x_1}{2}\right)\tanh\left(\frac{x_2}{2}\right)$ gives a non-vanishing contribution. We thus need to solve

\begin{equation}
    I_X =\frac{1}{4}I_X^- = \frac{1}{4}\iint d\omega_1d\omega_2\frac{1}{\omega_1 + E_3 + i0^+}\cdot\frac{\gamma^-_2\gamma^-_1}{\omega_1 + \omega_2 + E_2 + i0^+}\cdot\frac{1}{\omega_1+E_1+i0^+}.
\end{equation}

Using the Matsubara expansion we get (see Ref. (4))

\begin{equation}
\begin{split}
    I_X^-=&2\int d\omega_1 \frac{\psi\left( \frac{1}{2}-i\frac{E_2+\eta_2\mu_2+\omega_1}{2\pi T_2}\right)-\psi\left( \frac{1}{2}-i\frac{E_3+\eta_2\mu_2}{2\pi T_2}\right)}{\omega_1 + E_2-E_3+i0^+}\cdot\frac{1}{\omega_1+E_1}\tanh\left(\frac{\omega_1-\eta_1\mu_1}{2T_1} \right) = \\ 
    -i8T_1\pi\sum_{n=0}^{k_D=\frac{D}{2\pi T_2}-\frac{1}{2}}&\frac{\psi\left ( \frac{1}{2}+\frac{T_1}{T_2}\left(n+\frac{1}{2} \right)-i\frac{E_2+\eta_1\mu_1+\eta_2\mu_2}{2\pi T_2} \right ) - \psi\left( \frac{1}{2} -i\frac{E_3+\eta_2\mu_2}{2\pi T_2} \right )}{2\pi T_2(n+1/2)-i(E_2-E_3+\eta_1\mu_1)}\cdot\frac{1}{2\pi T_2 (n+1/2)-i(E_1+\eta_1\mu_1)}.
    \end{split}
\end{equation}
\\
When employing the Ozaki expansion the expression instead reads

\begin{equation}
I_X^- =-i8T_1\pi\sum_{\alpha>0} \frac{\psi\left ( \frac{1}{2} + \frac{b_\alpha T_1}{2\pi T_2}-i\frac{E_2+\eta_1\mu_1+ \eta_2\mu_2}{2\pi T_2} \right ) - \psi\left ( \frac{1}{2} - i\frac{E_3-\eta_2\mu_2}{2\pi T_2}\right ) }{b_\alpha T_1 + i(E_3-E_2-\mu_1)}\cdot\frac{R_\alpha}{b_\alpha T_1 - i(E_1+\eta_1\mu_1)}.
\end{equation}
\\
After accounting for the Keldysh indices we end up with $I_X = p_2p_1\frac{1}{4}I_X^-$.
\\
\\
Neither of the implementations for the direct and exchange integral ($T_1\ne T_2$) in QmeQ make the assumption that the integral needs to be purely imaginary in order for the matrix elements to be real. That requirement is enforced after the integral is evaluated. In fact, if the product of tunnel amplitudes in Eq.~\ref{Eq:W2Leijnse} has a finite complex part the integrals will be evaluated using the Ozaki approach. This also means that these general solutions can be used for future extensions to QmeQ where the RTD language is used for off-diagonal density matrices.
%-----------------------------------------------
\section{Parallelization and memory}

The cython version of the RTD approach is parallelized using the OpenMP standard. The idea behind the parallelization is that the kernels can be built row-by-row or column-by-column (and sometimes element-by-element, but that is not used). Therefore, each parallel thread can build e.g. a row in a kernel by calling for example \verb!ApproachRTD.generate_row_1st_order_kernel()!. The number of threads $p_t$ used by the program is $p_t=\min(p_{max},\ N)$ where $N$ is the number of many body states and $p_{max}$ is maximum the number of logical cores of the processor. When building $W_{dn}$ and $L_{nn}$ the number of threads is instead $p_{t,o}=\min(p_{max},\ N')$ where $N'$ is the number of off-diagonal elements included in the calculations. 
\\
\\
This approach to parallelization works perfectly fine for all kernels but $W_{dd}^{(2)}$ where additional care is needed since the usage the of symmetries in Eqs. \ref{Eq:W2symmetries1}-\ref{Eq:W2symmetries4} means that each call to \verb!ApproachRTD.generate_col_diag_kern_2nd_order()! can modify up to four columns, which can lead to race conditions. As a solution to this problem, all threads have their own copy of $W_{dd}^{(2)}$, and the result is reduced (summed) after all threads have finished their work load. $W_{dd}^{(2)}$ is thus stored in a numpy array of size $p_t \times N_{leads} \times N \times N$. The first slice of this is later used to store the full $W_{dd} = W_{dd}^{(2)}[0,:,:,:]$.
\\
\\
Giving each thread access to its own $N_{leads} \times N \times N$ array of course increases memory consumption, but it also increases execution speed as the symmetries can be fully utilized. Besides, second oder approaches are more computationally heavy than first order approaches and are thus normally used for smaller system sizes where memory is not a problem.
\\
\\
The total number of matrix elements stored in the RTD approach is

\begin{equation}
    N^2  N_{leads} (p_t + 2 ) + 2N'N N_{leads} + N^2 + N'.
\end{equation}
\\
This means that the required memory scales worse with number of single particle states $n$ than in the Pauli approach, but typically better than the other first order approaches. As an example, when using 8 threads the memory consumption scales roughly the same as the 1vN approach when using the $S^2$ symmetry (see Fig. 9 in Ref.~\cite{kirvsanskas2017qmeq}). Further memory optimizations is possible by combining $W_{E,1}$ and $W_{E,2}$ into a single kernel, reducing the first term to $ N^2  N_{leads} (p_t + 1 ) $.
\\
\\
In the python implementation parallelization is not intended to be used, and thus $p_t=1$. 
%---------------------------------------------Section:References
%\section{References}
\bibliographystyle{unsrt}
\bibliography{refs}

\end{document}